% Options for packages loaded elsewhere
\PassOptionsToPackage{unicode}{hyperref}
\PassOptionsToPackage{hyphens}{url}
%
\documentclass[
]{article}
\usepackage{amsmath,amssymb}
\usepackage{iftex}
\ifPDFTeX
  \usepackage[T1]{fontenc}
  \usepackage[utf8]{inputenc}
  \usepackage{textcomp} % provide euro and other symbols
\else % if luatex or xetex
  \usepackage{unicode-math} % this also loads fontspec
  \defaultfontfeatures{Scale=MatchLowercase}
  \defaultfontfeatures[\rmfamily]{Ligatures=TeX,Scale=1}
\fi
\usepackage{lmodern}
\ifPDFTeX\else
  % xetex/luatex font selection
\fi
% Use upquote if available, for straight quotes in verbatim environments
\IfFileExists{upquote.sty}{\usepackage{upquote}}{}
\IfFileExists{microtype.sty}{% use microtype if available
  \usepackage[]{microtype}
  \UseMicrotypeSet[protrusion]{basicmath} % disable protrusion for tt fonts
}{}
\makeatletter
\@ifundefined{KOMAClassName}{% if non-KOMA class
  \IfFileExists{parskip.sty}{%
    \usepackage{parskip}
  }{% else
    \setlength{\parindent}{0pt}
    \setlength{\parskip}{6pt plus 2pt minus 1pt}}
}{% if KOMA class
  \KOMAoptions{parskip=half}}
\makeatother
\usepackage{xcolor}
\usepackage[margin=1in]{geometry}
\usepackage{color}
\usepackage{fancyvrb}
\newcommand{\VerbBar}{|}
\newcommand{\VERB}{\Verb[commandchars=\\\{\}]}
\DefineVerbatimEnvironment{Highlighting}{Verbatim}{commandchars=\\\{\}}
% Add ',fontsize=\small' for more characters per line
\usepackage{framed}
\definecolor{shadecolor}{RGB}{248,248,248}
\newenvironment{Shaded}{\begin{snugshade}}{\end{snugshade}}
\newcommand{\AlertTok}[1]{\textcolor[rgb]{0.94,0.16,0.16}{#1}}
\newcommand{\AnnotationTok}[1]{\textcolor[rgb]{0.56,0.35,0.01}{\textbf{\textit{#1}}}}
\newcommand{\AttributeTok}[1]{\textcolor[rgb]{0.13,0.29,0.53}{#1}}
\newcommand{\BaseNTok}[1]{\textcolor[rgb]{0.00,0.00,0.81}{#1}}
\newcommand{\BuiltInTok}[1]{#1}
\newcommand{\CharTok}[1]{\textcolor[rgb]{0.31,0.60,0.02}{#1}}
\newcommand{\CommentTok}[1]{\textcolor[rgb]{0.56,0.35,0.01}{\textit{#1}}}
\newcommand{\CommentVarTok}[1]{\textcolor[rgb]{0.56,0.35,0.01}{\textbf{\textit{#1}}}}
\newcommand{\ConstantTok}[1]{\textcolor[rgb]{0.56,0.35,0.01}{#1}}
\newcommand{\ControlFlowTok}[1]{\textcolor[rgb]{0.13,0.29,0.53}{\textbf{#1}}}
\newcommand{\DataTypeTok}[1]{\textcolor[rgb]{0.13,0.29,0.53}{#1}}
\newcommand{\DecValTok}[1]{\textcolor[rgb]{0.00,0.00,0.81}{#1}}
\newcommand{\DocumentationTok}[1]{\textcolor[rgb]{0.56,0.35,0.01}{\textbf{\textit{#1}}}}
\newcommand{\ErrorTok}[1]{\textcolor[rgb]{0.64,0.00,0.00}{\textbf{#1}}}
\newcommand{\ExtensionTok}[1]{#1}
\newcommand{\FloatTok}[1]{\textcolor[rgb]{0.00,0.00,0.81}{#1}}
\newcommand{\FunctionTok}[1]{\textcolor[rgb]{0.13,0.29,0.53}{\textbf{#1}}}
\newcommand{\ImportTok}[1]{#1}
\newcommand{\InformationTok}[1]{\textcolor[rgb]{0.56,0.35,0.01}{\textbf{\textit{#1}}}}
\newcommand{\KeywordTok}[1]{\textcolor[rgb]{0.13,0.29,0.53}{\textbf{#1}}}
\newcommand{\NormalTok}[1]{#1}
\newcommand{\OperatorTok}[1]{\textcolor[rgb]{0.81,0.36,0.00}{\textbf{#1}}}
\newcommand{\OtherTok}[1]{\textcolor[rgb]{0.56,0.35,0.01}{#1}}
\newcommand{\PreprocessorTok}[1]{\textcolor[rgb]{0.56,0.35,0.01}{\textit{#1}}}
\newcommand{\RegionMarkerTok}[1]{#1}
\newcommand{\SpecialCharTok}[1]{\textcolor[rgb]{0.81,0.36,0.00}{\textbf{#1}}}
\newcommand{\SpecialStringTok}[1]{\textcolor[rgb]{0.31,0.60,0.02}{#1}}
\newcommand{\StringTok}[1]{\textcolor[rgb]{0.31,0.60,0.02}{#1}}
\newcommand{\VariableTok}[1]{\textcolor[rgb]{0.00,0.00,0.00}{#1}}
\newcommand{\VerbatimStringTok}[1]{\textcolor[rgb]{0.31,0.60,0.02}{#1}}
\newcommand{\WarningTok}[1]{\textcolor[rgb]{0.56,0.35,0.01}{\textbf{\textit{#1}}}}
\usepackage{graphicx}
\makeatletter
\def\maxwidth{\ifdim\Gin@nat@width>\linewidth\linewidth\else\Gin@nat@width\fi}
\def\maxheight{\ifdim\Gin@nat@height>\textheight\textheight\else\Gin@nat@height\fi}
\makeatother
% Scale images if necessary, so that they will not overflow the page
% margins by default, and it is still possible to overwrite the defaults
% using explicit options in \includegraphics[width, height, ...]{}
\setkeys{Gin}{width=\maxwidth,height=\maxheight,keepaspectratio}
% Set default figure placement to htbp
\makeatletter
\def\fps@figure{htbp}
\makeatother
\setlength{\emergencystretch}{3em} % prevent overfull lines
\providecommand{\tightlist}{%
  \setlength{\itemsep}{0pt}\setlength{\parskip}{0pt}}
\setcounter{secnumdepth}{-\maxdimen} % remove section numbering
\ifLuaTeX
  \usepackage{selnolig}  % disable illegal ligatures
\fi
\IfFileExists{bookmark.sty}{\usepackage{bookmark}}{\usepackage{hyperref}}
\IfFileExists{xurl.sty}{\usepackage{xurl}}{} % add URL line breaks if available
\urlstyle{same}
\hypersetup{
  pdftitle={Ανάλυση Δεδομένων Διαμαντιών},
  pdfauthor={Τσολακίδης},
  hidelinks,
  pdfcreator={LaTeX via pandoc}}

\title{Ανάλυση Δεδομένων Διαμαντιών}
\author{Τσολακίδης}
\date{3/4/2024}

\begin{document}
\maketitle

{
\setcounter{tocdepth}{2}
\tableofcontents
}
\hypertarget{ux3b1ux3c3ux3baux3b7ux3c3ux3b7-1.-ux3b1ux3bdux3acux3bbux3c5ux3c3ux3b7-ux3b4ux3b5ux3b4ux3bfux3bcux3adux3bdux3c9ux3bd-ux3b4ux3b9ux3b1ux3bcux3b1ux3bdux3c4ux3b9ux3ceux3bd}{%
\section{Ασκηση 1. Ανάλυση Δεδομένων
Διαμαντιών}\label{ux3b1ux3c3ux3baux3b7ux3c3ux3b7-1.-ux3b1ux3bdux3acux3bbux3c5ux3c3ux3b7-ux3b4ux3b5ux3b4ux3bfux3bcux3adux3bdux3c9ux3bd-ux3b4ux3b9ux3b1ux3bcux3b1ux3bdux3c4ux3b9ux3ceux3bd}}

Σε αυτή την ανάλυση θα μελετήσουμε το dataset διαμαντιών που περιέχεται
στο πακέτο ggplot2.

\hypertarget{ux3c6ux3ccux3c1ux3c4ux3c9ux3c3ux3b7-ux3c4ux3bfux3c5-dataset}{%
\subsection{Φόρτωση του
Dataset}\label{ux3c6ux3ccux3c1ux3c4ux3c9ux3c3ux3b7-ux3c4ux3bfux3c5-dataset}}

\begin{Shaded}
\begin{Highlighting}[]
\CommentTok{\# Φορτώνουμε τη βιβλιοθήκη ggplot2 για τη δημιουργία γραφημάτων}
\FunctionTok{library}\NormalTok{(ggplot2)}

\CommentTok{\# Αποθηκεύουμε τις στήλες \textquotesingle{}carat\textquotesingle{} και \textquotesingle{}price\textquotesingle{} σε ξεχωριστές μεταβλητές}
\NormalTok{carat }\OtherTok{\textless{}{-}}\NormalTok{ diamonds}\SpecialCharTok{$}\NormalTok{carat}
\NormalTok{price }\OtherTok{\textless{}{-}}\NormalTok{ diamonds}\SpecialCharTok{$}\NormalTok{price}

\CommentTok{\# Υπολογίζουμε τη συσχέτιση μεταξύ των μεταβλητών \textquotesingle{}carat\textquotesingle{} και \textquotesingle{}price\textquotesingle{}}
\NormalTok{correlation }\OtherTok{\textless{}{-}} \FunctionTok{cor}\NormalTok{(carat, price)}

\CommentTok{\# Εκτυπώνουμε τη συσχέτιση}
\FunctionTok{print}\NormalTok{(}\FunctionTok{paste}\NormalTok{(}\StringTok{"Correlation between carat and price: "}\NormalTok{, correlation))}
\end{Highlighting}
\end{Shaded}

\begin{verbatim}
## [1] "Correlation between carat and price:  0.921591301193477"
\end{verbatim}

\begin{Shaded}
\begin{Highlighting}[]
\CommentTok{\# Δημιουργούμε ένα scatterplot των μεταβλητών \textquotesingle{}carat\textquotesingle{} και \textquotesingle{}price\textquotesingle{} με χρήση της ggplot2}
\FunctionTok{ggplot}\NormalTok{(diamonds, }\FunctionTok{aes}\NormalTok{(}\AttributeTok{x =}\NormalTok{ carat, }\AttributeTok{y =}\NormalTok{ price)) }\SpecialCharTok{+}
  \FunctionTok{geom\_point}\NormalTok{(}\AttributeTok{color =} \StringTok{"blue"}\NormalTok{,  }\CommentTok{\# Χρώμα σημείων}
             \AttributeTok{shape =} \DecValTok{17}\NormalTok{,       }\CommentTok{\# Τύπος σημείων (τετράγωνα)}
             \AttributeTok{size =} \DecValTok{1}\NormalTok{,         }\CommentTok{\# Μέγεθος σημείων}
             \AttributeTok{alpha =} \FloatTok{0.7}\NormalTok{) }\SpecialCharTok{+}    \CommentTok{\# Διαφάνεια σημείων}
  \FunctionTok{labs}\NormalTok{(}\AttributeTok{x =} \StringTok{"Carat"}\NormalTok{, }\AttributeTok{y =} \StringTok{"Price"}\NormalTok{, }\AttributeTok{title =} \StringTok{"Scatterplot of carat vs price"}\NormalTok{)  }\CommentTok{\# Ορίζουμε τις ετικέτες των αξόνων και τον τίτλο του γραφήματος}
\end{Highlighting}
\end{Shaded}

\includegraphics{main_files/figure-latex/unnamed-chunk-1-1.pdf}

Στις περισσότερες περιπτώσεις όσο αυξάνονται τα καράτια αυξάνεται και η
τιμή, χωρίς αυτό να είναι απόλυτο

\begin{quote}
``Be alone, that is the secret of invention; be alone, that is when
ideas are born.''\\
- Nikola Tesla
\end{quote}

\hypertarget{ux3b1ux3c3ux3baux3b7ux3c3ux3b7-2.-ux3b1ux3bdux3acux3bbux3c5ux3c3ux3b7-ux3c4ux3bfux3c5-dataset-airquality}{%
\section{Ασκηση 2. Ανάλυση του dataset
`airquality'}\label{ux3b1ux3c3ux3baux3b7ux3c3ux3b7-2.-ux3b1ux3bdux3acux3bbux3c5ux3c3ux3b7-ux3c4ux3bfux3c5-dataset-airquality}}

\hypertarget{ux3c0ux3bfux3b9ux3ac-ux3b5ux3afux3bdux3b1ux3b9-ux3b7-ux3bcux3adux3c3ux3b7-ux3c4ux3b9ux3bcux3ae-ux3c4ux3b7ux3c2-ux3b8ux3b5ux3c1ux3bcux3bfux3baux3c1ux3b1ux3c3ux3afux3b1ux3c2-ux3b3ux3b9ux3b1-ux3c4ux3b7-ux3b4ux3b5ux3b4ux3bfux3bcux3adux3bdux3b7-ux3c0ux3b5ux3c1ux3afux3bfux3b4ux3bf}{%
\subsubsection{Ποιά είναι η μέση τιμή της θερμοκρασίας για τη δεδομένη
περίοδο;}\label{ux3c0ux3bfux3b9ux3ac-ux3b5ux3afux3bdux3b1ux3b9-ux3b7-ux3bcux3adux3c3ux3b7-ux3c4ux3b9ux3bcux3ae-ux3c4ux3b7ux3c2-ux3b8ux3b5ux3c1ux3bcux3bfux3baux3c1ux3b1ux3c3ux3afux3b1ux3c2-ux3b3ux3b9ux3b1-ux3c4ux3b7-ux3b4ux3b5ux3b4ux3bfux3bcux3adux3bdux3b7-ux3c0ux3b5ux3c1ux3afux3bfux3b4ux3bf}}

\begin{Shaded}
\begin{Highlighting}[]
\NormalTok{  mean\_temp }\OtherTok{\textless{}{-}} \FunctionTok{mean}\NormalTok{(airquality}\SpecialCharTok{$}\NormalTok{Temp)}
  \FunctionTok{cat}\NormalTok{(}\StringTok{"Η Μέση τιμή θερμοκρασίας είναι:"}\NormalTok{, mean\_temp, }\StringTok{"Fahrenheit }\SpecialCharTok{\textbackslash{}n}\StringTok{"}\NormalTok{)}
\end{Highlighting}
\end{Shaded}

\begin{verbatim}
## Η Μέση τιμή θερμοκρασίας είναι: 77.88235 Fahrenheit
\end{verbatim}

\hypertarget{ux3c0ux3bfux3b9ux3ac-ux3b7ux3bcux3adux3c1ux3b1-ux3aeux3c4ux3b1ux3bd-ux3b7-ux3b8ux3b5ux3c1ux3bcux3ccux3c4ux3b5ux3c1ux3b7}{%
\subsubsection{Ποιά ημέρα ήταν η
θερμότερη;}\label{ux3c0ux3bfux3b9ux3ac-ux3b7ux3bcux3adux3c1ux3b1-ux3aeux3c4ux3b1ux3bd-ux3b7-ux3b8ux3b5ux3c1ux3bcux3ccux3c4ux3b5ux3c1ux3b7}}

\begin{Shaded}
\begin{Highlighting}[]
\NormalTok{  hotterst\_day }\OtherTok{\textless{}{-}}\NormalTok{ airquality}\SpecialCharTok{$}\NormalTok{Day[}\FunctionTok{which.max}\NormalTok{(airquality}\SpecialCharTok{$}\NormalTok{Temp)]}
\NormalTok{  hotterst\_month }\OtherTok{\textless{}{-}}\NormalTok{ airquality}\SpecialCharTok{$}\NormalTok{Month[}\FunctionTok{which.max}\NormalTok{(airquality}\SpecialCharTok{$}\NormalTok{Temp)]}
  \FunctionTok{cat}\NormalTok{(}\StringTok{"Η ημέρα με την υψηλότερη θερμοκρασία ήταν η "}\NormalTok{, hotterst\_day, }\StringTok{"/"}\NormalTok{, hotterst\_month, }\StringTok{"}\SpecialCharTok{\textbackslash{}n}\StringTok{"}\NormalTok{)}
\end{Highlighting}
\end{Shaded}

\begin{verbatim}
## Η ημέρα με την υψηλότερη θερμοκρασία ήταν η  28 / 8
\end{verbatim}

\hypertarget{ux3c0ux3bfux3b9ux3ac-ux3b5ux3afux3c7ux3b5-ux3c4ux3bfux3bd-ux3c0ux3bfux3bbux3cd-ux3b1ux3adux3c1ux3b1}{%
\subsubsection{Ποιά είχε τον πολύ
αέρα;}\label{ux3c0ux3bfux3b9ux3ac-ux3b5ux3afux3c7ux3b5-ux3c4ux3bfux3bd-ux3c0ux3bfux3bbux3cd-ux3b1ux3adux3c1ux3b1}}

\begin{Shaded}
\begin{Highlighting}[]
\NormalTok{  windiest\_day }\OtherTok{\textless{}{-}}\NormalTok{ airquality}\SpecialCharTok{$}\NormalTok{Day[}\FunctionTok{which.max}\NormalTok{(airquality}\SpecialCharTok{$}\NormalTok{Wind)]}
\NormalTok{  windiest\_month }\OtherTok{\textless{}{-}}\NormalTok{ airquality}\SpecialCharTok{$}\NormalTok{Month[}\FunctionTok{which.max}\NormalTok{(airquality}\SpecialCharTok{$}\NormalTok{Wind)]}
  \FunctionTok{cat}\NormalTok{(}\StringTok{"Η ημέρα με τον περισσότερο αέρα ήταν η"}\NormalTok{, windiest\_day, }\StringTok{"/"}\NormalTok{, windiest\_month, }\StringTok{"}\SpecialCharTok{\textbackslash{}n}\StringTok{"}\NormalTok{)}
\end{Highlighting}
\end{Shaded}

\begin{verbatim}
## Η ημέρα με τον περισσότερο αέρα ήταν η 17 / 6
\end{verbatim}

\hypertarget{ux3c0ux3bfux3b9ux3adux3c2-ux3b7ux3bcux3adux3c1ux3b5ux3c2-ux3b7-ux3b8ux3b5ux3c1ux3bcux3bfux3baux3c1ux3b1ux3c3ux3afux3b1-ux3aeux3c4ux3b1ux3bd-ux3bcux3b5ux3b3ux3b1ux3bbux3cdux3c4ux3b5ux3c1ux3b7-ux3b1ux3c0ux3cc-90-ux3b2ux3b1ux3b8ux3bcux3bfux3cdux3c2-fahrenheit}{%
\subsubsection{Ποιές ημέρες η θερμοκρασία ήταν μεγαλύτερη από 90 βαθμούς
Fahrenheit?}\label{ux3c0ux3bfux3b9ux3adux3c2-ux3b7ux3bcux3adux3c1ux3b5ux3c2-ux3b7-ux3b8ux3b5ux3c1ux3bcux3bfux3baux3c1ux3b1ux3c3ux3afux3b1-ux3aeux3c4ux3b1ux3bd-ux3bcux3b5ux3b3ux3b1ux3bbux3cdux3c4ux3b5ux3c1ux3b7-ux3b1ux3c0ux3cc-90-ux3b2ux3b1ux3b8ux3bcux3bfux3cdux3c2-fahrenheit}}

\begin{Shaded}
\begin{Highlighting}[]
\NormalTok{  days\_over\_90 }\OtherTok{\textless{}{-}}\NormalTok{ airquality}\SpecialCharTok{$}\NormalTok{Day[}\FunctionTok{which}\NormalTok{(airquality}\SpecialCharTok{$}\NormalTok{Temp }\SpecialCharTok{\textgreater{}} \DecValTok{90}\NormalTok{)]}
\NormalTok{  days\_over\_90\_month }\OtherTok{\textless{}{-}}\NormalTok{ airquality}\SpecialCharTok{$}\NormalTok{Month[}\FunctionTok{which}\NormalTok{(airquality}\SpecialCharTok{$}\NormalTok{Temp }\SpecialCharTok{\textgreater{}} \DecValTok{90}\NormalTok{)]}
  \ControlFlowTok{for}\NormalTok{ (i }\ControlFlowTok{in} \DecValTok{1}\SpecialCharTok{:}\FunctionTok{length}\NormalTok{(days\_over\_90))\{}
    \FunctionTok{cat}\NormalTok{(days\_over\_90[i], }\StringTok{"/"}\NormalTok{, days\_over\_90\_month[i], }\StringTok{", "}\NormalTok{)}
\NormalTok{  \}}
\end{Highlighting}
\end{Shaded}

\begin{verbatim}
## 11 / 6 , 12 / 6 , 8 / 7 , 9 / 7 , 14 / 7 , 10 / 8 , 28 / 8 , 29 / 8 , 30 / 8 , 31 / 8 , 1 / 9 , 2 / 9 , 3 / 9 , 4 / 9 ,
\end{verbatim}

\begin{Shaded}
\begin{Highlighting}[]
\CommentTok{\# Get basic statistics about temperature}
\NormalTok{temp\_length }\OtherTok{\textless{}{-}} \FunctionTok{length}\NormalTok{(airquality}\SpecialCharTok{$}\NormalTok{Temp)}

\CommentTok{\# Create a sequence of days for the time series plot}
\NormalTok{days }\OtherTok{\textless{}{-}} \FunctionTok{seq}\NormalTok{(}\DecValTok{1}\NormalTok{, temp\_length)}

\CommentTok{\# Plot the time series of temperature}
\FunctionTok{plot}\NormalTok{(days, airquality}\SpecialCharTok{$}\NormalTok{Temp, }\AttributeTok{xlab =} \StringTok{"Day/Month {-} (0 = 1/5, 153 = 31/9)"}\NormalTok{, }\AttributeTok{ylab =} \StringTok{"Temperature"}\NormalTok{,}\AttributeTok{main =} \StringTok{"Temperature over Time"}\NormalTok{)}
\end{Highlighting}
\end{Shaded}

\includegraphics{main_files/figure-latex/unnamed-chunk-6-1.pdf}

Οι υψηλότερες θερμοκρασίες είναι τους μήνες 7-8

\begin{Shaded}
\begin{Highlighting}[]
\CommentTok{\# Create a boxplot of temperature}
\FunctionTok{boxplot}\NormalTok{(airquality}\SpecialCharTok{$}\NormalTok{Temp, }\AttributeTok{main =} \StringTok{"Temperature Distribution"}\NormalTok{, }\AttributeTok{ylab =} \StringTok{"Temperature"}\NormalTok{)}
\end{Highlighting}
\end{Shaded}

\includegraphics{main_files/figure-latex/unnamed-chunk-7-1.pdf}

\begin{Shaded}
\begin{Highlighting}[]
\CommentTok{\# Create a histogram of temperature}
\FunctionTok{hist}\NormalTok{(airquality}\SpecialCharTok{$}\NormalTok{Temp, }\AttributeTok{main =} \StringTok{"Temperature Histogram"}\NormalTok{, }\AttributeTok{xlab =} \StringTok{"Temperature"}\NormalTok{, }\AttributeTok{ylab =} \StringTok{"Frequency"}\NormalTok{, }\AttributeTok{col =} \StringTok{"skyblue"}\NormalTok{)}
\end{Highlighting}
\end{Shaded}

\includegraphics{main_files/figure-latex/unnamed-chunk-7-2.pdf}

\begin{Shaded}
\begin{Highlighting}[]
\CommentTok{\# calbulate the mean per month}
\NormalTok{avg\_temp\_per\_month }\OtherTok{\textless{}{-}} \FunctionTok{tapply}\NormalTok{(airquality}\SpecialCharTok{$}\NormalTok{Temp, airquality}\SpecialCharTok{$}\NormalTok{Month, mean)}

\CommentTok{\# Sort average temperatures and months}
\NormalTok{sorted\_avg\_temp }\OtherTok{\textless{}{-}}\NormalTok{ avg\_temp\_per\_month[}\FunctionTok{order}\NormalTok{(avg\_temp\_per\_month)]}

\CommentTok{\# Create a bar chart of average temperature per month}
\FunctionTok{barplot}\NormalTok{(sorted\_avg\_temp, }\AttributeTok{main =} \StringTok{"Average Temperature per Month"}\NormalTok{,}
        \AttributeTok{xlab =} \StringTok{"Month"}\NormalTok{, }\AttributeTok{ylab =} \StringTok{"Average Temperature"}\NormalTok{,}
        \AttributeTok{col =} \FunctionTok{rev}\NormalTok{(}\FunctionTok{heat.colors}\NormalTok{(}\FunctionTok{length}\NormalTok{(sorted\_avg\_temp))))}
\end{Highlighting}
\end{Shaded}

\begin{figure}
\centering
\includegraphics{main_files/figure-latex/gg-oz-gapminder-1.pdf}
\caption{Ο 8ος μήνας φαίνεται και ο θερμότερος}
\end{figure}

\begin{quote}
``Life is like riding a bicycle. To keep your balance, you must keep
moving.''\\
- Albert Einstein
\end{quote}

\end{document}
